% =============================================================================
\section{Xilinx ISE 14.7 Virtual Machine Setup}
% =============================================================================
%
The Xilinx instructions for installing the ISE 14.7 Virtual Machine (VM) can
be found in UG1227 \emph{ISE 14.7 VM for Windows 10 User Guide}~\cite{Xilinx_UG1227_2020}.
The following instructions \emph{do not} match the Xilinx installation guide
exactly.
%
\begin{enumerate}
\item \textbf{Install Oracle VirtualBox}

Oracle VirtualBox can be downloaded from
\href{https://www.virtualbox.org}{https://www.virtualbox.org}.

Download and install VirtualBox. Then use \emph{File$\rightarrow$Check for Updates},
and download and install the latest Extension Pack.
VirtualBox version 7.0.8 was the latest version on 7/8/2024. 

The Xilinx ISE 14.7 VM was also used successfully under VirtualBox version 6.1.40.
Xilinx UG1227 indicates the VM was originally developed using VirtualBox 5.2
(p11~\cite{Xilinx_UG1227_2020}).

\item \textbf{Extract the ISE 14.7 Virtual Machine Image}

This is the step where these instructions differ from Xilinx UG1227. The Xilinx
recommended installation method includes Windows 10 wrapper applications that 
offer no real benefit to users already familiar with VirtualBox and VMs. 
Rather than use the Xilinx installation method, this step extracts the VM from 
within the Xilinx zip file, and imports it directly into VirtualBox.

\begin{enumerate}
\item Unzip the 15.5GB file \verb+Xilinx_ISE_14.7_Win10_14.7_VM_0213_1.zip+
\newline 
The zip file can be unzipped onto a USB drive (eg., a 1TB Samsung T7 SSD).
Unzipping takes a few minutes.
\item The Open Virtualization Archive (OVA) is located in the unzipped directory
\newline
\verb+Xilinx_ISE_14.7_Win10_14.7_VM_0213_1\ova\14.7_VM.ova+
\item Start VirtualBox
\item Use VirtualBox \emph{File$\rightarrow$Import Appliance} to import the OVA file.
\item The default VM name is \verb+ISE_14.7_VM_base+. Rename it \verb+ISE_14.7_VM+.
\item (Optional) Change the \emph{Machine Base Folder}. The imported VM requires 39.7GB. 
The machine base folder can be located on an external USB SSD. The VM is imported 
into the folder \verb+ISE_14.7_VM+, so the machine base folder should not 
include the VM name.
\item Click \emph{Finish} to import the VM. Importing takes a few minutes.
\item The VirtualBox then displays a new \verb+ISE_14.7_VM+ entry, with the
virtual machine in the \verb+Powered Off+ state.
\end{enumerate}
%
\item \textbf{Configure the Virtual Machine Settings}
%
\begin{enumerate}
\item Click on the VirtualBox \emph{Settings} button to access the VM settings.
\item The \emph{System}, \emph{Motherboard}, settings show the VM is configured
for 6144MB. Increase this if possible, eg., on a 64GB machine, allocate 32GB.
\item The \emph{System}, \emph{Processor}, settings show the VM is configured
for 1 CPU. \emph{Do not increase the number of CPUs} as the VM will hang on
boot (see the note below). 
\item The \emph{Display}, \emph{Screen}, settings show the VM display controller
is set to VBoxVGA with 3D acceleration disabled. There is a warning message at
the bottom of the display indicating \verb+Invalid settings detected+, and clicking
on the warning displays a message indicating the screen setting is not the 
recommended VMSVGA. \emph{Do not change the display setting.}
\item The \emph{Network}, \emph{Adapter 1}, settings show the VM is configured
for the \verb+Host-only Adapter+. Depending on your needs, this may need to be
changed. UG1227 has additional details.
\item The \emph{USB} settings show the VM is configured to use USB 1.1.
Change this to USB 2.0.
\end{enumerate}
%
\newpage
\item \textbf{Configure Oracle Linux}
%
\begin{enumerate}
\item Click on the VirtualBox \emph{Start} button to boot the ISE 14.7 VM.
\item The GUI displays a red progress bar on the bottom-left side, and the
red text\newline \textcolor{red}{\texttt{Oracle Linux Server 6.4}} on the bottom-right side.
Booting the VM takes less than a minute. The red progress bar and text eventually turn white,
and then the Linux graphical desktop appears.

\textcolor{magenta}{This is where the VM hangs if the VM is configured for more than 1 CPU.}

\item Add a Terminal to the top-panel of the Linux desktop by using the mouse to navigate
to (but do not left-mouse-click) \emph{Applications$\rightarrow$System Tools$\rightarrow$Terminal}, 
then right-mouse-click, and select \emph{Add this launcher to panel}.

\item Click on the Terminal to start it, and then investigate the operating system settings:
%
\begin{verbatim}
[ise@localhost ~]$ cat /etc/issue
Oracle Linux Server release 6.4
Kernel \r on an \m

[ise@localhost ~]$ cat /etc/redhat-release 
Red Hat Enterprise Linux Server release 6.4 (Santiago)

[ise@localhost ~]$ uname -a
Linux localhost.localdomain 2.6.32-358.el6.x86_64 #1 SMP Fri Feb 22 
13:35:02 PST 2013 x86_64 x86_64 x86_64 GNU/Linux
\end{verbatim}
%$
The terminal output shows that Oracle Linux Server 6.4 is a release based on 
Red Hat Enterprise Linux (RHEL) release 6.4, and that the Linux kernel is 
version 2.6 from 2013.

The terminal output also shows the default user name is \verb+ise+ and the default 
machine name is \verb+localhost+.

\item Update the VirtualBox Guest Additions used in the Oracle Linux VM.
\begin{itemize}
\item Select the VirtualBox menu option \emph{Devices$\rightarrow$Insert 
Guest Additions CD image}.
\item A CD image appears on the Linux desktop with the name \verb+VBox_GAs_7.0.8+\newline
(the suffix is based on the VirtualBox version). 
\item Click in the VM and the CD autorun dialog appears.
\item Click \emph{Ok}, and then \emph{Run}.
\item Enter the root password \verb+xilinx+ to run the Guest Additions installer.
\item Press \verb+Return+ in the terminal to close it once
the installation completes.
\item Select the CD image, right-mouse-click, and select \emph{Eject}.
\item Type the command \verb+sudo shutdown -h now+ to shutdown the VM
\item Re-start the VM
\end{itemize}
%
This VM shutdown and re-start is needed for the VM to use the new Guest Additions.

\item Set the Linux Time Zone per UG1227 instructions (see pp13-14~\cite{Xilinx_UG1227_2020}).

The Linux clock on the top-right should update to match your host time.

\item Set the display resolution.

Xilinx recommends a minimum screen resolution of 1280 $\times$ 1024 (p9~\cite{Xilinx_UG1227_2020}).

From the menu at the top of the Linux desktop, select \emph{System$\rightarrow$Preferences$\rightarrow$Display}. Select a screen resolution from the pull-down menu. If you resize the VM window,
a new (non-standard resolution) entry will appear in the menu, select that entry.

\newpage
\item Change the host name


\begin{verbatim}
[ise@localhost ~]$ sudo su
[ise@localhost ~]# vi /etc/sysconfig/network
NETWORKING=yes
HOSTNAME=localhost.localdomain

# Change the last line to
HOSTNAME=xilinx-ise-14_7-vm
	
[ise@localhost ~]# service network restart

# Fails with an eth0 error

[ise@localhost ~]# shutdown -h now
\end{verbatim}
%$
\textcolor{red}{\bf TODO: Understand why network restart did not work!}

Restart the VM and the terminal prompt changes to:
%
\begin{verbatim}
[ise@xilinx-ise-14_7-vm ~]$
\end{verbatim}
%$
Note that the host name must not include a period, since periods are used for domain names.
This is why the underscore was used in the ISE version number.

%\newpage
\item Change the default user name. \textcolor{red}{\bf WORK-IN-PROGRESS!}

The default user \verb+ise+ can be renamed, but the user \verb+ise+ cannot be logged in
when this is done. The VM can be started as the root user as follows:
%
\begin{itemize}
\item Shut down the VM
\item Interrupt the boot process by clicking in the VM window and pressing a key to
enter the GRUB boot menu
\item Type \verb+e+ to edit the boot command
\item Navigate down to the kernel entry and type \verb+e+ to edit the kernel command line
\item Delete the word \verb+quiet+ and enter the word \verb+single+ (for single-user boot)
\item Hit enter to save the changes, and then \verb+b+ to boot with the new kernel command line
\item Linux will boot to a text command line with the root prompt
\item Rename the user and home directories to your own name, eg.,
%
\begin{verbatim}
[ise@localhost ~]$ sudo su
[ise@localhost ~]# usermod -l dhawkins ise
[ise@localhost ~]# ls /home
ise
[ise@localhost ~]# usermod -d /home/dhawkins -m dhawkins
[ise@localhost ~]# ls /home
dhawkins
[ise@localhost ~]# shutdown -h now
\end{verbatim}
%$
\end{itemize}
%
Restart the VM and the system hangs after the red progress bar completes.

Using these same commands to rename \verb+dhawkins+ and the home directory back to those of \verb+ise+ restores booting into Linux.

\item \textcolor{blue}{\bf Setup shared drive}

Configure the VM for shared access to \verb+C:/+ (\verb+C_DRIVE+) and then mount it within
the VM to \verb+/mnt/c/+. This mounting scheme will be familiar to users of Windows 
Subsystem for Linux (WSL).

\end{enumerate}


Did I get the Digilent driver to work? Did I have to install Adept into the VM?

p22: USB - the VM is setup for Xilinx and Digilent cables.

The Genesys USB port appears as "Xilinx USB Cable" and it is a USB 2.0 device.

Didn't the Genesys User Guide say there were two USB programmer settings? How do I change devices?

Actually, start Vivado and see if the hardware manager recognizes the cable.

Vivado connects to \verb+localhost/xilinx_tcf/Xilinx/00000000000000+, and it has a
device property of jsn-DLC9LP-00000000000000 operating at 6MHz. 

The JTAG programmer finds a single device, xc5vlx50t with a JTAG IDCODE of 0xC2A96093.


\end{enumerate}

